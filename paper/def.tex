%%%%-----Language Name-------%%%%
\newcommand*\Coq{\text{Coq}}
\newcommand*\Caml{\text{OCaml}}
%%%%------Reference-------%%%%%%
\newcommand*\figref[1]{Figure~\ref{fig:#1}}
\newcommand*\secref[1]{Section~\ref{sec:#1}}

%%%%-----Source Language-------%%%%
\newcommand{\leaksem}{$\rightarrow_{leak}$}




%%%%----Comment-----%%%%%
%\newcommand{\sptodo}[1] {\textbf{\textcolor{blue}{sp:}}\textbf{\textcolor{blue}{#1}}}
\newcommand\vlnote[1]{{\sffamily\color{blue!50!red}[VL] #1}}
\newcommand{\gb}[1]{{\sffamily\color{blue!70!red}[GB] #1}}
\newcommand{\bg}[1]{{\sffamily\color{blue!90!red}[BG] #1}}
\newcommand{\spnote}[1]{{\sffamily\color{green!40!red}[SP] #1}}
\newcommand{\bash}[1]{{\sffamily\color{blue!40!red}[BAS] #1}}

%%%%%%%%%%%%%%
\newcommand{\expr}{\mathsf{Expr}}
\newcommand{\val}{\mathsf{Value}}
\newcommand{\const}{\mathsf{Const}}

\newcommand{\kw}[1]{{\ensuremath{\mathsf{#1}}}}
\newcommand{\kskip}{\kw{skip}}
\newcommand{\kif}{\kw{if}}
\newcommand{\klet}{\kw{let}}
\newcommand{\kin}{\kw{in}}
\newcommand{\krun}{\kw{run}}
\newcommand{\kthen}{\kw{then}}
\newcommand{\kelse}{\kw{else}}
\newcommand{\kunit}{\kw{unit}}
\newcommand{\kheap}{\Theta}
\newcommand{\kvm}{\Pi}
\newcommand{\effectlabel}{\kw{effect_l}}



\newcommand{\iseq}[2]{{#1}; {#2}}
\newcommand{\eass}[2]{\ensuremath{#1  :=#2}}
\newcommand{\cond}[3]{\kif\ {#1}\ \kthen\ {#2}\ \kelse\ {#3}}
\newcommand{\run}[1]{\krun\ {#1}}
\newcommand{\hpe}[2]{\kw{hp}\ {#1}.{#2}}
\newcommand{\alloc}[2]{\&{(#1 : #2)}}
\newcommand{\deref}[2]{^^21{(#1 : #2)}}
\newcommand{\addr}[2]{{#1 : #2}}
\newcommand{\app}[3]{#1 \ \overline{#2}^{#3}}
\newcommand{\abs}[4]{\lambda \overline{#1 : #2}^#3.#4} 
\newcommand{\elet}[3]{\klet\ #1 := #2 \ \kin \ {#3}}

%%%%Types 
\newcommand{\funt}[4]{\overline{#1}^#2 \rightarrow #3, #4}
\newcommand{\reft}[2]{\kw{ref}(#1, #2)}


%%%%Evaluation
\newcommand{\eval}[4]{#1,#2 \downarrow #3,#4}

%%%%%Typing 
\newcommand{\typerule}[5]{#1,#2 \vdash #3 : #4, #5}









%%%%% Math %%%%%%
\DeclareUnicodeCharacter{00B7}{\cdot}
\DeclareUnicodeCharacter{00D7}{\ifmmode\times\else\texttimes\fi}
\DeclareUnicodeCharacter{00F7}{\ifmmode\div\else\textdiv\fi}
\DeclareUnicodeCharacter{03B3}{\ensuremath{\gamma}}
\DeclareUnicodeCharacter{03BA}{\ensuremath{\kappa}}
\DeclareUnicodeCharacter{03C4}{\ensuremath{\tau}}
\DeclareUnicodeCharacter{03C6}{\ensuremath{\varphi}}
\DeclareUnicodeCharacter{2081}{\ensuremath{{}_1}}
\DeclareUnicodeCharacter{2082}{\ensuremath{{}_2}}
\DeclareUnicodeCharacter{2113}{\ensuremath{\ell}}
\DeclareUnicodeCharacter{21A6}{\ensuremath{\mapsto}}
\DeclareUnicodeCharacter{21D2}{\implies}
\DeclareUnicodeCharacter{21D3}{\Downarrow}
\DeclareUnicodeCharacter{2200}{\forall}
\DeclareUnicodeCharacter{2203}{\exists}
\DeclareUnicodeCharacter{2208}{\in}
\DeclareUnicodeCharacter{2218}{\ensuremath{\circ}}
\DeclareUnicodeCharacter{2227}{\mathop{\land}}
\DeclareUnicodeCharacter{222A}{\cup}
\DeclareUnicodeCharacter{2260}{\neq}
\DeclareUnicodeCharacter{2264}{\leqslant}
\DeclareUnicodeCharacter{2291}{\sqsubseteq}


%%% Semantics %%%%%%%
\newcommand{\rname}[1]{\ensuremath{[\textsc{#1}]}}
\let\xinferrule\inferrule
\renewcommand{\inferrule}[3][]{%
	\ifx&#1&
	\xinferrule*{#2}{#3}
	\else
	\xinferrule*[right=\rname{#1}]{#2}{#3}
	\fi}
\newcommand{\subst}[2]{#1 := #2}
\newcommand{\sem}[2]{\llbracket #1 \rrbracket #2}

\def\substitute#1by#2{[#2/#1]}

\newcommand*\vtrue{\ensuremath{{t\kern-.4ex t}}}
\newcommand*\vfalse{\ensuremath{{f\kern-.7ex f}}}


\newtheorem{lemma}{Lemma}






\newcommand{\Equiv}[4]{{#1} \thicksim {#2}~:~ {#3} \Longrightarrow {#4}}  
\newcommand{\pre}{\varphi}
\newcommand{\post}{\psi}

\newcommand{\ecmodele}[1]{\{#1\}}
\newcommand{\ecmodel}[1]{[{#1}]}
\newcommand{\eqon}[1]{=_{\{#1\}}}



\DeclarePairedDelimiter\floor{\lfloor}{\rfloor}
\newcommand{\vleft}{\{1\}}
\newcommand{\vright}{\{2\}}



\newtheorem{prop}{Proposition}
\newtheorem{lem}{Lemma}
\newtheorem{thm}{Theorem}
\theoremstyle{definition}
\newtheorem{defn}{Definition}
\newtheorem{rem}{Remark}